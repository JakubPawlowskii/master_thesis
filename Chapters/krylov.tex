\chapter{Lancz\H{o}s based numerical methods for quantum many-body systems}
\thispagestyle{chapterBeginStyle}

One of the two purposes of this thesis is to develop and test a set of numerical tools based on the Lancz\H{o}s algorithm,
which is an iterative method for finding extremal eigenvalues of a large, sparse matrices. Therefore, this chapter serves as a
pedagogical introduction to the core ideas of these methods, however without going too deep into details of algorithm analysis.
In the exposition we follow the excellent treatments found in~\textcite{Sandvik2010} and PhD thesis by~\textcite{Crivelli2016}.
The interested and mathematically inclined reader is referred to the classic textbook of numerical linear algebra by~\textcite{Trefethen1997}.

In the first part of this chapter, the Lancz\H{o}s algorithm is developed in its simplest form which in itself is useful for efficient calculation
of a ground state eigenvalue and eigenvector of sparse Hamiltonians, and thus for determining their ground state properties.
However, in this work we are mainly interested in infinite temperature correlation functions, which in principle require sampling of the whole spectrum.
To this end, in subsequent sections we develop a scheme for time evolution of arbitrary state, called the Krylov propagator~\autocite{Park1986},
 and combine it with the idea of Dynamical Quantum Typicality (DQT), which states that a single pure state can have the same
 properties as an ensemble density matrix~\autocite{Gemmer2003,Goldstein2006,Popescu2006}.
We finish this chapter with a proposal of employing this method to the identification of local integrals of motion (\textcolor{red}{cite my bachelors})

\section{Lanczos method for ground state calculation}
