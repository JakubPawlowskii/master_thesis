\chapter{Lancz{o}s based numerical methods for quantum many-body systems}
\thispagestyle{chapterBeginStyle}

One of the two purposes of this thesis is to develop and test a set of numerical tools based on the Lancz{o}s algorithm,
which is an iterative method for finding extremal eigenvalues of large, sparse matrices. Therefore, this chapter serves as a
pedagogical introduction to the core ideas of these methods, however without going too deep into details of algorithm analysis.
In the exposition we follow the excellent treatments found in~\textcite{Sandvik2010} and PhD thesis by~\textcite{Crivelli2016}.
The interested and mathematically inclined reader is referred to the classic textbook of numerical linear algebra by~\textcite{Trefethen1997}.

The first part of this chapter is devoted to the Lancz{o}s algorithm in its simplest form, that is useful for efficient calculation
of extremal eigenvalues and eigenvectors of sparse Hamiltonians, in particular (the) ground state, thus for determining their
ground state properties.
However, in this work we are mainly interested in infinite temperature calculations, for which in principle sampling of the whole
spectrum is required.
To this end, in subsequent sections we develop a scheme for time evolution of arbitrary state, called the Krylov propagator~\autocite{Park1986},
 and combine it with the idea of Dynamical Quantum Typicality (DQT), which states that a single pure state can have the same
 properties as an ensemble density matrix~\autocite{Gemmer2003,Goldstein2006,Popescu2006}.
We finish this chapter with a proposal of employing this method to the identification of local integrals of motion in a given
tight-binding system.
 (\textcolor{red}{cite my bachelors})

 For the remainder of this chapter, let \(H\) denote arbitrary tight-binding Hamiltonian, \(\mathcal{H}\) the associated Hilbert space,
 and \(\dimension = \textrm{dim}\left(\mathcal{H}\right)\) its dimension.


\section{Lanczos method for ground state calculation}

\subsection{Problems with Exact Diagonalization}

The most straightforward numerical method for studying discrete quantum many-body systems is without a doubt
Exact Diagonalization (ED)~\autocite{Weisse2008}.
It allows one to obtain numerically exact set of eigenvalues and eigenvectors and subsequently compute any desired properties
of the system, be it thermal expectation values, time evolution, Green's functions etc. Unfortunately, the starting point of any
ED calculation is the expression of the Hamiltonian as a dense matrix, in the Hilbert space basis of choice. Taking into account
the fact that the dimension many-body Hilbert space grows exponentially with the size of the system, the memory cost quickly becomes
prohibitive, even when exploiting conservation laws and related symmetries. For example, in the case of a spin chain of length L, with 
on-site basis dimension being 2, the full dimension of the Hilbert space would be \(2^{L}\). Taking a modest length of 25 sites, that gives
\(2^{25} = 33554432\approx 3.36 \cdot 10^7\) basis states and a memory footprint of Hamiltonian matrix of around 9PB (using double-precision
floating point numbers), which is 9000 times more than the typical consumer hard drive capacity of 1TB. Even assuming some kind of distributed
memory platform allowing for handling such large matrices, the computational complexity of ED, requiring \(O(\dimension^3)\) operations,
is the next major hurdle. Therefore, it is exceedingly
difficult to probe the thermodynamic limit physics and ED calculations suffer from finite size effects.

Closer investigation of the Hamiltonian matrix, expressed in computational basis (cf. eq ...) quickly reveals the inefficiency of dense
storage. Looking at Figure~(\textcolor{red}{Here figure with Hamiltonian, basis ordered by magnetization}), we see that most of
the matrix elements are zero. In fact only about \(\dimension_0\propto \dimension \) out of \(\dimension^2\) matrix elements
are non-zero. Hence, a numerical scheme leveraging this sparsity is highly desirable. This is exactly what the Lanczos
algorithm does.

\subsection{Implementation of Lanczos algorithm}

