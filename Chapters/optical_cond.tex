\chapter{Optical conductivity in spin chains\label{app:opt_cond}}
\thispagestyle{chapterBeginStyle}

% \textcite{Sirker2011,Sirker2020} for derivation of \(\sigma(\omega)\) from Kubro formula
% \textcite{Gohmann2022} for the derivation of Kubo formula

The aim of this appendix is to derive the Kubo formula for the optical conductivity in spin chains,
specifically the spectral representation of the real part, as given in Eq.~\eqref{eq:kubo}.
In the spirit of the pedagogical approach of this thesis, we shall present all the steps in detail.
% The derivation follows closely the one presented in Ref.~\autocite{Gohmann2022}. 
For simplicity, we consider the nearest-neighbor version of the Heisenberg model:
\begin{equation}
    H =  \underbrace{\frac{J}{2} \sum_{\ell=1}^{L} \left(\Sp_{\ell} \Sm_{\ell+1} + \Sm_{\ell}\Sp_{\ell+1} \right)}_{H_{0}}
    + \underbrace{J \Delta  \sum_{\ell=1}^{L} \Sz_{\ell}\Sz_{\ell+1}}_{H_{\text{int}}}
    - h \sum_{\ell=1}^{L} \Sz_{\ell}
    \label{eq:nn_heisenberg}
\end{equation}

\section{Peierls phase}
To discuss the linear \textit{response}, we need to couple our model to some external field, in our
case the electromagnetic field. Under the Jordan-Wigner transformation, the Hamiltonian~\eqref{eq:nn_heisenberg}
maps to a model of spinless fermions, with a density-density interaction term.
\begin{equation}
    H_{\text{F} } = \frac{J}{2}\sum_{\ell=1}^{L} \left( c_{\ell}^{\dagger} c_{\ell+1} + c_{\ell+1}^{\dagger} c_{\ell} \right)
    + J \Delta \sum_{\ell=1}^{L} \left(n_{\ell} - \frac{1}{2}\right)\left(n_{\ell+1}-\frac{1}{2}\right)
    - h \sum_{\ell=1}^{L} \left(n_{\ell} - \frac{1}{2}\right)
\end{equation}
where \(n_{\ell} = c^{\dag}_{\ell}c_{\ell}\).
In this picture, the kinetic part of the fermionic Hamiltonian couple to the electromagnetic field
via the so-called \textbf{Peierls phase}, which is a manifestation of the \(U(1)\) gauge field. To understand
how this happens in the formalism of second quantization, we need to return for a moment back to the
first quantization picture for spinless fermions. This section is based on the book by~\textcite{Essler2005}.
A single particle Hamiltonian in a 3-dimensional space is given by
\begin{equation}
    h_{1} = \frac{\mathbf{p}^{2}}{2m} + V(\mathbf{x} )
\end{equation}
where \(\mathbf{p}\) is the momentum operator and \(V(x)\) is some external potential. In the presence of an
electromagnetic 4-potential \(\left(\Phi(\mathbf{x},t ), A^{\alpha }(\mathbf{x} ,t)/c\right)\),
the momentum operator is modified, via the minimal coupling, to the following form:
\begin{equation}
    h_{1} \to \frac{1}{2m}\left(\mathbf{p} + \frac{e}{c}\mathbf{A}\right)^{2} - e\,\Phi(\mathbf{x},t) + V(\mathbf{x} )
\end{equation}
where \(\Phi \) is the scalar potential and \(\mathbf{A}\) is the vector potential.
Moreover, we can choose the \textbf{radiation gauge} \(\Phi(\mathbf{x} ,t) = 0\).
Now, let us proceed with the second quantization procedure in the standard fashion. Let
\(c^{\dag}_{\ell},c_{\ell}\) create and annihilate a particle in the Wannier state \(\phi(\mathbf{x - \mathbf{R}_{\ell}})\),
corresponding to the \(\ell\)-th lattice site (we assume only a single band).
Written using those operators, the single-particle part of the Hamiltonian \textit{without} the electromagnetic field is
\begin{equation}
    H_{1} = \sum_{\ell,\ell^{\prime} } t_{\ell\ell^{\prime}} c^{\dag}_{\ell}c_{\ell^{\prime}}
\end{equation}
where the hopping matrix elements are given by the matrix elements of the single-particle operator,
between the Wannier states:
\begin{equation}
    t_{\ell\ell^{\prime}} = \int \dd{x^3} \phi^{*}(\mathbf{x} - \mathbf{R}_{\ell})
    \left(\frac{\mathbf{p}^{2}}{2m} + V(\mathbf{x} )\right)\phi(\mathbf{x} - \mathbf{R}_{\ell^{\prime}})
\end{equation}
From the above expression, we can see that the influence of the electromagnetic field enters the second-quantized
Hamiltonian via the hopping integrals. Let us see now how this happens
\begin{align}
    t_{\ell\ell^{\prime}} & \to \int \dd{x^3} \phi^{*}(\mathbf{x} - \mathbf{R}_{\ell})
    \left[\frac{1}{2m}\left( \mathbf{p} + \frac{e}{c} \mathbf{A}  \right)^2  + V(\mathbf{x} )\right]
    \phi(\mathbf{x} - \mathbf{R}_{\ell^{\prime}}) \nonumber                                                                 \\
                          & \triangleq \int \dd{x^3} \phi^{*}(\mathbf{x} - \mathbf{R}_{\ell}) \mathrm{e}^{-i e \lambda / c}
    \left[\frac{1}{2m}\left( \mathbf{p} - \frac{e}{c} \left( \mathbf{A} - \grad{\lambda } \right)   \right)^2  + V(\mathbf{x} )\right]
    \mathrm{e}^{i e \lambda / c} \phi(\mathbf{x} - \mathbf{R}_{\ell^{\prime}})
\end{align}
Note, that the equality denoted with \(\triangleq \) is valid for any differentiable, complex gauge field \(\lambda (\mathbf{x},t)\).
We choose it to be
\begin{equation}
    \lambda (\mathbf{x},t) =  \int_{\mathbf{x}_0 }^{\mathbf{x}} \dd{\mathbf{x}^{\prime}} \cdot \mathbf{A}(\mathbf{x}^{\prime},t)
\end{equation}
for an arbitrary reference point \(\mathbf{x}_0 \). Furthermore, we introduce modified Wannier functions
\begin{equation}
    \Tilde{\phi}(\mathbf{x} - \mathbf{R}_{\ell}) = \mathrm{e}^{i e \lambda(\mathbf{x} ,t) / c} \phi(\mathbf{x} - \mathbf{R}_{\ell})
\end{equation}
This way, we can write the modified hopping integrals as
\begin{equation}
    {t}_{\ell\ell^{\prime}}(t) = \int \dd{x^3} \Tilde{\phi}^{*}(\mathbf{x} - \mathbf{R}_{\ell})
    \left[\frac{\mathbf{p}^2}{2m}  + V(\mathbf{x} )\right]
    \Tilde{\phi}(\mathbf{x} - \mathbf{R}_{\ell^{\prime}})
\end{equation}
thus recovering (almost) the original form of the hopping integral, but with the modified Wannier functions.
We need a few more steps to get to the final result. First, let us assume that the Wannier functions
are strongly localized, and that the vector potential varies slowly on the scale of the lattice constant.
This permits the approximation
\begin{equation}
    \Tilde{\phi}(\mathbf{x - \mathbf{R}_{\ell}}) \approx \phi(\mathbf{x - \mathbf{R}_{\ell}}) \mathrm{e}^{i e \lambda(\mathbf{R}_{\ell} ,t) / c}
\end{equation}
We can also assume that the hopping integrals beyond nearest neighbors are negligible, and denote
its nearest-neighbor value by \(t\). Then, we can write the single-particle Hamiltonian as
\begin{equation}
    H_1(t) = -t \sum_{\langle \ell,\ell^{\prime} \rangle } \mathrm{e}^{-i e \left( \lambda(\mathbf{R}_{\ell},t) -
    \lambda (\mathbf{R}_{\ell^{\prime}},t) \right) /c } c^{\dag}_{\ell}c_{\ell^{\prime}}
\end{equation}
Phase appearing in the above expression is precisely the Peierls phase. This is a general result, valid for
any lattice geometry and in the tight-binding approximation. However, in this thesis we are only interested in a
one-dimensional, periodic lattice, so let us simplify a bit. We can slightly modify the gauge field
and absorb the constant \(\frac{e}{c}\)
\begin{align}
    \lambda_{\ell,\ell+1}(t) & = \frac{e}{c} \int_{\mathbf{R}_{\ell}}^{\mathbf{R}_{\ell+1}} \dd{\mathbf{x}} \cdot \mathbf{A}(\mathbf{x},t)
    = \frac{e}{c} \int_{\mathbf{x}_0}^{\mathbf{R}_{\ell+1}} \dd{\mathbf{x}} \cdot \mathbf{A}(\mathbf{x},t) + \frac{e}{c} \int_{\mathbf{R}_{\ell}}^{\mathbf{x}_0} \dd{\mathbf{x}} \cdot \mathbf{A}(\mathbf{x},t) \nonumber               \\
                             & = \frac{e}{c} \int^{\mathbf{R}_{\ell+1}}_{\mathbf{x}_0} \dd{\mathbf{x}}\cdot \mathbf{A}(\mathbf{x},t)- \frac{e}{c} \int_{\mathbf{x}_0}^{\mathbf{R}_{\ell}} \dd{\mathbf{x}}\cdot \mathbf{A}(\mathbf{x},t)
    = \frac{e}{c}\left[ \lambda(\mathbf{R}_{\ell+1},t) - \lambda(\mathbf{R}_{\ell},t) \right]
    \label{eq:peierls_phase}
\end{align}
and finally obtain the desired result
\begin{equation}
    H_1(t) = -t \sum_{ \ell=1}^{L} \mathrm{e}^{i \lambda_{\ell,\ell+1}(t) } c^{\dag}_{\ell}c_{\ell+1} + \mathrm{h.c.}
    \label{eq:peierls_hamiltonian}
\end{equation}
Before we return to the original problem, let us see what is the relation of \(\lambda_{\ell,\ell+1}(t)\)  to the electric field.
We define the electric field in terms of the 4-potential as
\begin{equation}
    \mathbf{E}(\mathbf{x},t) = -\grad{\phi(\mathbf{x},t)} - \frac{1}{c} \pdv{\mathbf{A}(\mathbf{x},t)}{t}
    = -\frac{1}{c} \pdv{\mathbf{A} (\mathbf{x},t)}{t}
\end{equation}
where the gradient of scalar potential is zero, since we are working in the radiation gauge.
As we consider a one-dimensional, periodic system, i.e. ring lattice, the electric field vector can be decomposed into
the radial and azimuthal components \(\mathbf{E} (\mathbf{x} ,t) = -E(r,t)\mathbf{e}_{\varphi}\), where
\(\mathbf{e}_{\varphi }\) is the unit vector along the ring and \(r\) is the distance from the center of the ring.
Thus, we can choose the vector potential to have the same form \(\mathbf{A}(\mathbf{x},t) = A(r,t)\mathbf{e}_{\varphi}\).
Then, we can carry out the integration in \eqref{eq:peierls_phase} and obtain
\begin{equation}
    \lambda(t) \equiv \lambda_{\ell,\ell+1}(t) = \frac{e a_0}{c} A(R,t)
\end{equation}
where \(a_0 = 2\pi \frac{R}{L}\) is the lattice constant and \(R\) is the radius of the ring.

\section{Linear response of the spin current}

Now, we can return from the fermionic realm back to our spin chain. For the remainder of this section, we set
the speed of light \(c = 1\). Using new insight from the previous section and following~\textcite{Gohmann2022}, we can write our Heisenberg
Hamiltonian, with the Peierls phase, as
\begin{equation}
    H^{\prime}(t) = \underbrace{\frac{J}{2} \sum_{\ell=1}^{L} \left(\mathrm{e}^{i \lambda(t)} \Sp_{\ell} \Sm_{\ell+1} + \mathrm{e}^{-i \lambda(t)} \Sm_{\ell}\Sp_{\ell+1} \right)}_{H_{\lambda }(t)}
    + \underbrace{J \sum_{\ell=1}^{L} \Sz_{\ell}\Sz_{\ell+1}}_{H_{\text{int}}}
    - h \sum_{\ell=1}^{L} \Sz_{\ell}
\end{equation}
where the gauge field \(\lambda(t)\) is related to time-dependent electric field \(E(t)\) as
\begin{equation}
    \pdv{\lambda(t)}{t} = - e a_0 E(t)
\end{equation}
Thus, in Fourier space, we have the relation
\begin{equation}
    \lambda_{\mathrm{F}} (\omega) = \frac{e a_0}{i \omega} E_{\mathrm{F}} (\omega), \text{ where } \lambda_{\mathrm{F}} (\omega) = \int_{-\infty}^{\infty} \dd{t} \mathrm{e}^{i \omega t} \lambda(t)
\end{equation}
provided that the Fourier transform exists. We assume that the electric field is switched on adiabatically, and
is asymptotically constant. In terms of the gauge field, this can be expressed as
\(\abs{\lambda (t)} \leq \mathrm{e}^{\varepsilon t} \) for \(t\to -\infty \) and
\(\abs{\lambda (t)} \leq b t \) for \(t\to \infty \), where \(\varepsilon > 0\) and \(b>0\) are some constants.
Then, in the \( t \to \infty \) limit, we have \[\mathrm{e}^{i (\Re(\omega) t + i \Im(\omega) t )} \lambda (t)
    \leq \mathrm{e}^{i \Re (\omega) t}  \mathrm{e}^{- \Im (\omega)  t} b t,\]
whereas in the \(t \to -\infty \) limit,
we have \[\mathrm{e}^{i (\Re (\omega) t + i \Im (\omega) t )} \lambda (t) \leq \mathrm{e}^{i \Re (\omega) t}
    \mathrm{e}^{(- \Im (\omega) + \varepsilon )t}\]
Thus, the Fourier transform of the gauge field exists in the
vertical strip \(0 < \Im (\omega) < \varepsilon \).

Of course, application of such external field will induce a current in the system. We can obtain the
corresponding current operator by inspecting the nearest-neighbor restriction of the long-range
current~\eqref{eq:spin_current} and noticing that the gauge field enters the Hamiltonian only through
the Peierls phase, which is a scalar and commutes with the spin operators. Thus, the current density
reads
\begin{equation}
    \mathcal{J}^{\sigma }_{\ell}(t) = i\frac{J}{2} \left( \mathrm{e}^{i \lambda(t)} \Sp_{\ell} \Sm_{\ell+1} - \mathrm{e}^{-i \lambda(t)} \Sm_{\ell}\Sp_{\ell+1} \right)
\end{equation}
and the total current is
\begin{equation}
    \mathcal{J}^{\sigma } (t) = \sum_{\ell=1}^{L} J^{\sigma }_{\ell}(t)
    = i\frac{J}{2} \sum_{\ell=1}^{L} \left( \mathrm{e}^{i \lambda(t)} \Sp_{\ell} \Sm_{\ell+1} - \mathrm{e}^{-i \lambda(t)} \Sm_{\ell}\Sp_{\ell+1} \right)
    \label{eq:total_current}
\end{equation}
Here, as in \(H^{\prime} (t)\), the time argument is an explicit time-dependence, not the Heisenberg picture one.
Note that we reserve the script \(\mathcal{J}^{\sigma}\) for current including Peierls phase,
and keep the lowercase \(j^{\sigma }\) for the current without it.
To apply the reasoning typical for the linear response theory, we need to identify the perturbation
\(V(t)\), such that our time-dependent Hamiltonian reads is decomposed as \(H^{\prime} = H + V(t)\).
However, we will only need this perturbation up to the first order, so let us expand \(H_{\lambda }(t)\)  into the
power series in the gauge field \(\lambda(t)\) as
\begin{align}
    H_{\lambda }(t) & = \frac{J}{2} \sum_{\ell=1}^{L} \left[\mathrm{e}^{i \lambda(t)} \Sp_{\ell} \Sm_{\ell+1}
    + \mathrm{e}^{-i \lambda(t)} \Sm_{\ell}\Sp_{\ell+1} \right] \nonumber                                                                             \\
                    & = \frac{J}{2} \sum_{\ell=1}^{L} \left[\left( 1 + i \lambda (t) + O(\lambda ^2) \right)  \Sp_{\ell} \Sm_{\ell+1}
    + \left( 1 - i \lambda (t) + O(\lambda ^2) \right) \Sm_{\ell}\Sp_{\ell+1} \right] \nonumber                                                       \\
                    & = \frac{J}{2} \sum_{\ell=1}^{L} \left[ \Sp_{\ell} \Sm_{\ell+1} + \Sm_{\ell}\Sp_{\ell+1} \right] +
    \frac{i J}{2} \sum_{\ell=1}^{L} \left[ \Sp_{\ell} \Sm_{\ell+1} \lambda (t) - \Sm_{\ell}\Sp_{\ell+1} \lambda (t) \right] + O(\lambda ^2) \nonumber \\
                    & = H_0 + \lambda(t) j^{\sigma } + O(\lambda ^2)
\end{align}
Thus we have \(H^{\prime} = H + \lambda (t) j^{\sigma } + O(\lambda ^2)\) and the perturbation is
identified as, up to the first order, \(V(t) = \lambda (t) j^{\sigma }\).
Analogously, we can expand the current operator to obtain
\begin{equation}
    \mathcal{J}^{\sigma } (t) = j^{\sigma } - \lambda(t) H_0 + O(\lambda ^2)
\end{equation}

We have now all the ingredients to proceed with the calculation of linear response.
From now on, we suppress the explicit time-dependence of the operators, and assume all appearing time
indices to be related to time evolution. Despite the pedagogical character of this section, we will
refrain from the full derivation of the Kubo linear response theory, as this would enlarge this long
appendix even further. Instead, here we will only sketch the idea and present the final formula,
whereas the interested reader is, as usual, referred to the literature for details~\autocite{Mahan2000,Pottier2014,Gohmann2021}.
Formal solution of the Schrödinger equation, with the Hamiltonian \(H^{\prime} = H + V\), is given
by the unitary time evolution operator \(U(t,t_{0})\) which acting on the initial state \(\ket{\psi
    (t_{0})}\) yields the state at time \(t\), \(\ket{\psi (t)} = U(t,t_{0}) \ket{\psi (t_{0})}\).
Moreover, it fulfills the equation
\begin{equation}
    i \partial_t U(t,t_{0}) = H^{\prime} U(t,t_{0})
    \label{eq:time_evolution_operator}
\end{equation}
Let now \(\rho (t)\) be the density operator at time \(t\).
We assume that our system, before the perturbation is switched on at time \(t_0\), is in the thermal equilibrium
thus \(\rho (t<t_0 ) = \rho_0\), where \(\rho_0\) denotes the canonical ensemble density operator,
\( \rho_0 = \mathrm{e}^{-\beta H_0} / \Tr \mathrm{e}^{-\beta
    H_0}\), where \(\beta = 1 / k_B T\) is the inverse temperature. Corresponding average will be
denoted as \(\langle \dots  \rangle_0 \). By expanding the density operator in the time-evolved basis
of eigenstates of \(H\), the unperturbed Hamiltonian, we obtain
\begin{align}
    \rho (t) & = \sum_{n} \frac{\mathrm{e}^{-\beta E_n} }{\Tr \mathrm{e}^{-\beta H_0}} \ketbra{n(t)}{n(t)} = \sum_n \frac{\mathrm{e}^{-\beta E_n}}{\Tr \mathrm{e}^{-\beta H_0}} U(t,t_0) \ketbra{n}{n} U^{\dag}(t,t_0)\nonumber \\
             & = U(t,t_0) \rho_0 U^{\dag}(t,t_0)
\end{align}
The time-dependent average of any operator \(A\) is then
given using \(\rho (t)\),
\begin{equation}
    \langle A \rangle_t = \Tr \left( \rho (t) A \right) = \langle U^{\dag}(t,t_0)  A  U(t,t_0)\rangle_0
\end{equation}
where for the last equality we have used the cyclic property of the trace. Core goal of the linear response
theory is the calculation of the above quantity, up to linear order in the perturbation \(V\).
The standard approach is to move to the so-called \textbf{interaction picture}, where the time evolution
given by the unperturbed Hamiltonian is absorbed into the states and the observables, and the time evolution
operator depends only on the perturbation. It facilitates a systematic way for the perturbative solution
of the equation~\eqref{eq:time_evolution_operator}, yielding subsequent approximations to the time evolution
operator. Omitting the details, we present the final result for the time-dependent
average up to the first order in the perturbation,
\begin{align}
    \langle A \rangle_t & = \langle \mathrm{e}^{i H_0 t}  A \mathrm{e}^{-i H_0 t} \rangle_0
    - i \int_{-\infty }^{t} \mathrm{d}t^{\prime} \langle \comm*{\mathrm{e}^{i H_0 t} A \mathrm{e}^{-i H_0 t}}
    {\mathrm{e}^{i H_0 t^{\prime} }V\mathrm{e}^{-i H_0 t^{\prime} }} \rangle_0 \nonumber         \\
                        & = \langle \mathrm{e}^{i H_0 t}  A \mathrm{e}^{-i H_0 t} \rangle_0
    + \int_{-\infty }^{\infty } \mathrm{d}t^{\prime} \left(-i \theta (t-t^{\prime} ) \langle \comm*{\mathrm{e}^{i H_0 t} A \mathrm{e}^{-i H_0 t}}
    {\mathrm{e}^{i H_0 t^{\prime} }V\mathrm{e}^{-i H_0 t^{\prime} }} \rangle_0 \right) \nonumber \\
                        & =\langle \mathrm{e}^{i H_0 t}  A \mathrm{e}^{-i H_0 t} \rangle_0
    + \int_{-\infty }^{\infty } \mathrm{d}t^{\prime} \left(-i \theta (t-t^{\prime} ) \langle \comm*{\mathrm{e}^{i H_0 (t-t^{\prime} )} A \mathrm{e}^{-i H_0 (t-t^{\prime} )}}
    {V} \rangle_0 \right)
    \label{eq:linear_response}
\end{align}
known as the \textbf{Kubo formula}. The last equality comes from the fact that
\(\Tr(A(t)B(t^{\prime} )) = \Tr(A(t-t^{\prime} )B)\), which is a consequence of the cyclic property of the trace. 
We also assume that the perturbation is turned adiabatically,
starting from the infinite past, i.e. \(t_0 \to -\infty \). What is interesting about the above formula, is that
it does not require the knowledge of neither full evolution operator \(U(t,t_0)\) nor the evolution operator
generated by the perturbation.

Let us now take the current operator \(\mathcal{J}^{\sigma }\) as the observable \(A\), and
the perturbation \(V = \lambda j^{\sigma } + O(\lambda^2)\), remembering that \(\lambda \) is still time-dependent.