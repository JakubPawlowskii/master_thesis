\chapter{Two Fourier transforms\label{app:fourier}}
\thispagestyle{chapterBeginStyle}
The goal of this appending is the calculation of two Fourier transforms, which appear in the main text.
They originate from the definition of finite-time averaging~\eqref{eq:finite_time_avg} and
the integrated spectral function~\eqref{eq:integrated_spectral}. Let us begin with the latter one,
as it is simpler. We have
\begin{align}
    \mathcal{F}\left[\hs{A(t)}{B}\right] &= \lim _{\varepsilon \to 0^+} \frac{1}{2 \pi} \int_{-\infty}^{\infty} \mathrm{d} t 
    \; e^{i \omega t-|t| \varepsilon}\hs{A(t)}{B} = \lim _{\varepsilon \to 0^+} \frac{1}{2 \pi}
    \int_{-\infty}^{\infty} \mathrm{d} t \; e^{i \omega t-|t| \varepsilon} \frac{1}{\dimension}
    \Tr\left[\left(e^{iHt}Ae^{-iHt}\right)^{\dagger}B\right] \nonumber\\
    &=\lim _{\varepsilon \to 0^+} \frac{1}{2 \pi}
    \int_{-\infty}^{\infty} \mathrm{d} t \; e^{i \omega t-|t| \varepsilon} \frac{1}{\dimension}
    \Tr\left[e^{iHt}\left(\sum_{m} \ketbra{m}{m}\right) 
    A\left(\sum_{n} \ketbra{n}{n}\right)e^{-iHt}B\right] \nonumber\\
    &= \frac{1}{\dimension} \frac{1}{2 \pi}\sum_{n,m} \lim _{\varepsilon \to 0^+} 
    \int_{-\infty}^{\infty} \mathrm{d} t \; e^{i \omega t-|t| \varepsilon}
    \Tr\left[e^{i \epsilon_m t}\ketbra{m}{m} A\ketbra{n}{n}e^{-i \epsilon_n t}B\right] \nonumber\\
    &= \frac{1}{\dimension} \frac{1}{2 \pi}\sum_{n,m} A_{mn}  \lim _{\varepsilon \to 0^+} 
    \int_{-\infty}^{\infty} \mathrm{d} t \; e^{i \omega t-|t| \varepsilon}e^{i (\epsilon_m-\epsilon_n) t}
    \sum_{k} \underbrace{\braket{k}{m}}_{=\delta_{km}} \matrixel{n}{B}{k} \nonumber\\
    &= \frac{1}{\dimension} \frac{1}{2 \pi}\sum_{n,m} A_{mn} B_{nm}  \underbrace{\lim _{\varepsilon 
    \to 0^+} \int_{-\infty}^{\infty} \mathrm{d} t \; e^{i \omega t-|t| 
    \varepsilon}e^{i (\epsilon_m-\epsilon_n) t}}_{\mathcal{I}}\label{eq:spectral function simplified}
\end{align}
Let us massage it a bit further, by calculating the integral \(\mathcal{I}\) explicitly.
\begin{align}
\mathcal{I} &= \lim _{\varepsilon \to 0^+} \int_{-\infty}^{\infty} \mathrm{d} t \; 
e^{ i (\epsilon_m-\epsilon_n+\omega)t -|t| \varepsilon} = \lim _{\varepsilon \to 0^+} 
\Bigg[\lim_{T_1\to -\infty}\int_{T_1}^{0} \mathrm{d} t \; e^{ i (\epsilon_m-\epsilon_n+\omega -i \varepsilon)t}\nonumber\\
&+\lim_{T_2\to \infty}\int_{0}^{T_2} \mathrm{d} t \;  e^{ i (\epsilon_m-\epsilon_n+\omega + i\varepsilon)t}\Bigg]
= \lim _{\varepsilon \to 0^+} \Bigg[\lim_{T_1\to -\infty} 
\frac{1-e^{ i (\epsilon_m-\epsilon_n+\omega )T_1} e^{\varepsilon T_1}}{i (\epsilon_m-\epsilon_n+\omega -i \varepsilon)}\nonumber\\
 &+ \lim_{T_2\to \infty} \frac{e^{ i (\epsilon_m-\epsilon_n+\omega )T_2} e^{-\varepsilon T_2}-1}
{i (\epsilon_m-\epsilon_n+\omega +i \varepsilon)}\Bigg]
= \lim _{\varepsilon \to 0^+} \left[\frac{i}{\epsilon_n-\epsilon_m-\omega +i \varepsilon} + 
\frac{i}{\epsilon_m-\epsilon_n+\omega +i \varepsilon} \right]\nonumber\\
 &= \lim _{\varepsilon \to 0^+}
\frac{2\varepsilon}{(\epsilon_m-\epsilon_n+\omega -i \varepsilon)(\epsilon_m-\epsilon_n+\omega +i \varepsilon)}
=\lim _{\varepsilon \to 0^+} \frac{2\varepsilon}{(\epsilon_m-\epsilon_n+\omega)^2 +\varepsilon^2}
\end{align}
The obtained result is a limit of the so-called Poisson kernel. This happens to be
a representation of Dirac \(\delta\) in the form of a limit of a sequence of functions~\autocite{byron1992mathematics}
\begin{equation}
\lim _{\varepsilon \to 0^+} \frac{1}{\pi} \frac{\varepsilon}{x^2 +\varepsilon^2} = \delta(x) 
\end{equation}
Thus we get
\begin{equation}
\mathcal{I} = \lim _{\varepsilon \to 0^+} \frac{2\varepsilon}{(\epsilon_m-\epsilon_n+\omega)^2 +\varepsilon^2}
= 2\pi \delta(\epsilon_m-\epsilon_n+\omega)
\end{equation}
Inserting this result into equation~\eqref{eq:spectral function simplified} we arrive at
\begin{align}
\mathcal{F}\left[\hs{A(t)}{B}\right] &= \frac{1}{\dimension} \frac{1}{2 \pi}\sum_{n,m} A_{mn} B_{nm}  
\lim _{\varepsilon \to 0^+} \int_{-\infty}^{\infty} \mathrm{d} t \; e^{i \omega t-|t| 
\varepsilon}e^{i (\epsilon_m-\epsilon_n) t} \nonumber\\
=& \frac{1}{\dimension}\sum_{n,m} A_{mn} B_{nm} \delta(\epsilon_m-\epsilon_n+\Omega)
\end{align}
which is our desired results.