\chapter{Two Fourier transforms\label{app:fourier}}
\thispagestyle{chapterBeginStyle}
The goal of this appending is the calculation of two Fourier transforms, which appear in the main text.
They originate from the definition of finite-time averaging~\eqref{eq:finite_time_avg} and
the integrated spectral function~\eqref{eq:integrated_spectral}. 

\section{Fourier transform of the correlation function}
Let us begin with the latter one,
as it is simpler. We have, starting from the left-hand side of~\eqref{eq:fourier_corr}
\begin{align}
    \mathcal{F}\left[\hs{A(t)}{B}\right] & = \lim _{\varepsilon \to 0^+} \frac{1}{2 \pi} \int_{-\infty}^{\infty} \mathrm{d} t
    \; e^{i \omega t-|t| \varepsilon}\hs{A(t)}{B} = \lim _{\varepsilon \to 0^+} \frac{1}{2 \pi}
    \int_{-\infty}^{\infty} \mathrm{d} t \; e^{i \omega t-|t| \varepsilon} \frac{1}{\dimension}
    \Tr\left[\left(e^{iHt}Ae^{-iHt}\right)^{\dagger}B\right] \nonumber                                                                    \\
                                         & =\lim _{\varepsilon \to 0^+} \frac{1}{2 \pi}
    \int_{-\infty}^{\infty} \mathrm{d} t \; e^{i \omega t-|t| \varepsilon} \frac{1}{\dimension}
    \Tr\left[e^{iHt}\left(\sum_{m} \ketbra{m}{m}\right)
    A\left(\sum_{n} \ketbra{n}{n}\right)e^{-iHt}B\right] \nonumber                                                                        \\
                                         & = \frac{1}{\dimension} \frac{1}{2 \pi}\sum_{n,m} \lim _{\varepsilon \to 0^+}
    \int_{-\infty}^{\infty} \mathrm{d} t \; e^{i \omega t-|t| \varepsilon}
    \Tr\left[e^{i \epsilon_m t}\ketbra{m}{m} A\ketbra{n}{n}e^{-i \epsilon_n t}B\right] \nonumber                                          \\
                                         & = \frac{1}{\dimension} \frac{1}{2 \pi}\sum_{n,m} A_{mn}  \lim _{\varepsilon \to 0^+}
    \int_{-\infty}^{\infty} \mathrm{d} t \; e^{i \omega t-|t| \varepsilon}e^{i (\epsilon_m-\epsilon_n) t}
    \sum_{k} \underbrace{\braket{k}{m}}_{=\delta_{km}} \matrixel{n}{B}{k} \nonumber                                                       \\
                                         & = \frac{1}{\dimension} \frac{1}{2 \pi}\sum_{n,m} A_{mn} B_{nm}  \underbrace{\lim _{\varepsilon
        \to 0^+} \int_{-\infty}^{\infty} \mathrm{d} t \; e^{i \omega t-|t|
            \varepsilon}e^{i (\epsilon_m-\epsilon_n) t}}_{\mathcal{I}}\label{eq:spectral function simplified}
\end{align}
The \(\varepsilon \) part in the exponent is a trick to make the integral converge, without assuming anything
about the behavior of the correlation function at infinity.
Let us massage it a bit further, by calculating the integral \(\mathcal{I}\) explicitly.
\begin{align}
    \mathcal{I} & = \lim _{\varepsilon \to 0^+} \int_{-\infty}^{\infty} \mathrm{d} t \;
    e^{ i (\epsilon_m-\epsilon_n+\omega)t -|t| \varepsilon} = \lim _{\varepsilon \to 0^+}
    \Bigg[\lim_{T_1\to -\infty}\int_{T_1}^{0} \mathrm{d} t \; e^{ i (\epsilon_m-\epsilon_n+\omega -i \varepsilon)t}\nonumber         \\
                & +\lim_{T_2\to \infty}\int_{0}^{T_2} \mathrm{d} t \;  e^{ i (\epsilon_m-\epsilon_n+\omega + i\varepsilon)t}\Bigg]
    = \lim _{\varepsilon \to 0^+} \Bigg[\lim_{T_1\to -\infty}
    \frac{1-e^{ i (\epsilon_m-\epsilon_n+\omega )T_1} e^{\varepsilon T_1}}{i (\epsilon_m-\epsilon_n+\omega -i \varepsilon)}\nonumber \\
                & + \lim_{T_2\to \infty} \frac{e^{ i (\epsilon_m-\epsilon_n+\omega )T_2} e^{-\varepsilon T_2}-1}
    {i (\epsilon_m-\epsilon_n+\omega +i \varepsilon)}\Bigg]
    = \lim _{\varepsilon \to 0^+} \left[\frac{i}{\epsilon_n-\epsilon_m-\omega +i \varepsilon} +
    \frac{i}{\epsilon_m-\epsilon_n+\omega +i \varepsilon} \right]\nonumber                                                           \\
                & = \lim _{\varepsilon \to 0^+}
    \frac{2\varepsilon}{(\epsilon_m-\epsilon_n+\omega -i \varepsilon)(\epsilon_m-\epsilon_n+\omega +i \varepsilon)}
    =\lim _{\varepsilon \to 0^+} \frac{2\varepsilon}{(\epsilon_m-\epsilon_n+\omega)^2 +\varepsilon^2}
\end{align}
The obtained result is a limit of the so-called Poisson kernel. This happens to be
a representation of Dirac \(\delta\) in the form of a limit of a sequence of functions~\autocite{Byron1992}
\begin{equation}
    \lim _{\varepsilon \to 0^+} \frac{1}{\pi} \frac{\varepsilon}{x^2 +\varepsilon^2} = \delta(x)
\end{equation}
Thus we get
\begin{equation}
    \mathcal{I} = \lim _{\varepsilon \to 0^+} \frac{2\varepsilon}{(\epsilon_m-\epsilon_n+\omega)^2 +\varepsilon^2}
    = 2\pi \delta(\epsilon_m-\epsilon_n+\omega)
\end{equation}
Inserting this result into equation~\eqref{eq:spectral function simplified} we arrive at
\begin{align}
    \mathcal{F}\left[\hs{A(t)}{B}\right] & = \frac{1}{\dimension} \frac{1}{2 \pi}\sum_{n,m} A_{mn} B_{nm}
    \lim _{\varepsilon \to 0^+} \int_{-\infty}^{\infty} \mathrm{d} t \; e^{i \omega t-|t|
    \varepsilon}e^{i (\epsilon_m-\epsilon_n) t} \nonumber                                                                    \\
    =                                    & \frac{1}{\dimension}\sum_{n,m} A_{mn} B_{nm} \delta(\epsilon_m-\epsilon_n+\Omega)
\end{align}
which is our desired results.

\section{Extension of Fourier transform to \(L^2(\RR)\) and the Fourier transform of \(\frac{sin(x)}{x}\)}
Now, let us tackle the more difficult integral, necessary to show, that Eq.~\eqref{eq:finite_time_avg_integral}
and Eq~\eqref{eq:finite_time_avg} are equivalent. We start the usual way, by passing to spectral representation
of the observable \(A\)
\begin{equation}
    \bar{A}^{\tau} = \int_{-\infty}^{\infty} \mathrm{d}t A(t) \frac{\sin (t/\tau )}{\pi t}
    = \sum_{n,m}  \matrixel{n}{A}{m} \ketbra{n}{m} \frac{1}{\pi } \int_{-\infty}^{\infty} \mathrm{d}t\;
    \mathrm{e}^{i (E_n-E_m) t} \frac{\sin (t/\tau )}{t}
\end{equation}
where \(H\ket{n} = E_n \ket{n}\) is the eigenbasis of our Hamiltonian. Our problem reduces to calculating
the integral
\begin{equation}
    I =\mathcal{F}\left[ \frac{\sin (\alpha t)}{t} \right]=  \int_{-\infty}^{\infty} \mathrm{d}t\; \mathrm{e}^{i \omega  t} \frac{\sin (\alpha  t )}{t}
\end{equation}
where \(\alpha = 1/\tau\) and \(\omega = E_n-E_m\), which can be recognized as a Fourier transform of
\(\sin (\alpha  t )/t\). Here is where things become a bit subtle. The Fourier transform is traditionally
defined on functions belonging to the space \(L^1 \equiv L^1(\RR)\) of Lebesgue-measurable, \textbf{integrable
    functions}~\autocite{Rudin1987}, that is
\begin{equation}
    L^1 = \left\{f:\RR \to \CC \mid \int_{-\infty}^{\infty} \mathrm{d}x \; |f(x)| < \infty \right\}
\end{equation}
It is a member of a larger family of spaces \(L^p \equiv L^p(\RR),\; p \in \left[ 1,\infty  \right] \), defined as
\begin{equation}
    L^p = \left\{f:\RR \to \CC \mid \int_{-\infty}^{\infty} \mathrm{d}x \; |f(x)|^p < \infty \right\}
\end{equation}
thus guaranteeing the existence of respective \(L^p\) norms \(\Vert f \Vert_p = \left(\int_{-\infty}^{\infty} \mathrm{d}x \; |f(x)|^p \right)^{1/p}\).
Unfortunately, the function \(\sin (x)/x\) is not in \(L^1\). To see it, note that
\(\int_{-\infty}^{\infty} \mathrm{d}x\; \vert \sin (x)/x \vert = 2 \int_{0}^{\infty} \mathrm{d}x \vert \sin (x)/x \vert  \)
and consider the following
\begin{align*}
    \int_{\pi}^{(N+1)\pi}\mathrm{d}x\;  \left|\frac{\sin x}x\right| & =\sum_{k=1}^N\int_{k\pi}^{(k+1)\pi}\mathrm{d}x \;\left|\frac{\sin x}x\right|
    =\sum_{k=1}^N\int_0^{\pi}\mathrm{d}t \;\frac{|\sin(t+k\pi)|}{t+k\pi}
    =\sum_{k=1}^N\int_0^{\pi}\mathrm{d}t \;\frac{|\sin t|}{t+k\pi}                                                                                 \\\
                                                                    & \geq \sum_{k=1}^N\frac 1{(k+1)\pi}\int_0^{\pi}\mathrm{d}t\;  \sin t
    =\frac 2{\pi}\sum_{k=1}^N\frac 1{k+1},
\end{align*}
Thus, the integral is bounded from below by the harmonic series, which diverges.
Now, the question is whether we can somehow make sesne of the integral \(I\).
It turns out, that
even though the function is not integrable, it is \textbf{square integrable} --- that is, it belongs to the
space \(L^2\)
which can be seen by noting that
\begin{equation}
    \int_{-\infty }^{\infty} \mathrm{d}x \; \left|\frac{\sin x}x\right|^2  =
    2\left[ \int_{0}^{1} \mathrm{d}x \; \frac{\sin^2(x)}{x^2} + \int_{1}^{\infty } \mathrm{d}x \; \frac{\sin^2(x)}{x^2}   \right]
\end{equation}
and that the first integral is finite, while the second one is bounded from above by \(\int_{1}^{\infty}\mathrm{d}x\;  1/x^2 = 1\).
The space \(L^2\) is unique among all the \(L^p\) --- it is the only \textit{Hilbert space},
where the norm \(\Vert \cdot \Vert_p \) is induced by an inner product
\begin{equation}
    L^2 \cross L^2 \ni (f,g) \to \langle f,g \rangle = \int_{-\infty}^{\infty} \mathrm{d}x \; f(x)g^*(x) \in \CC
\end{equation}
\(g^*\) being the complex conjugate of \(g\). We are going to work out an extenstion of the Fourier transform
to the space \(L^2\), using a classic \textbf{density argument}. To this end, we need the following theorem
\begin{theorem}[Parseval-Plancherel]

    Let \(f,g \in L^1 \cap L^2\), and let \(\hat{f},\hat{g}\) be respective Fourier transforms. Then
    \begin{equation*}
        \langle f,g \rangle = \frac{1}{2\pi }\langle \hat{f} ,\hat{g}  \rangle  \; .
    \end{equation*}
\end{theorem}
The proof on this theorem is not difficult, but it would require us to introduce some new notions and deviate
too much from the scope of this thesis. Thus, we shall take it for granted and the reader is encouraged to
look it up in any textbook on Fourier analysis~\autocite{Rudin1987,Stein2011}. Now, we are ready for
\begin{theorem}
    \(L^1 \cap L^2\) is a dense subset of \(L^2\).
\end{theorem}
\begin{proof}
    We need to show, that for any \(f \in L^2\), there exists a sequence \(\{f_n\}_{n=1}^{\infty}\) of functions
    in \(L^1 \cap L^2\) such that \(\lim_{n \to \infty} \Vert f_n - f \Vert_2 = 0\). So let us take an arbitrary
    function \(f \in L^2\) and define a sequence of functions \(\{f_n\}_{n=1}^{\infty}\) as follows
    \begin{equation*}
        f_n(t) = \mathbb{1}_{[-n,n]}(t)f(t)
    \end{equation*}
    where \(\mathbb{1}_{[-n,n]}\) is the indicator function of the interval \([-n,n]\). Obviously, \(f_n \in L^2\)
    as \(\lVert f_n \rVert_2 \leq \lVert f \rVert_2 \leq \infty \).  We need to show that for every
    \(n \in \NN\setminus \left\{ 0 \right\}  \), \(f_n \in L^1\). Indeed, we have
    \begin{equation*}
        \lVert f_n \rVert_1 = \lVert \mathbb{1}_{[-n,n]}f \rVert_1 \leq \lVert \mathbb{1}_{[-n,n]} \rVert_2 \lVert f \rVert_2
        = \sqrt{2n} \lVert f \rVert_2 < \infty
    \end{equation*}
    where the first inequality above is an application of the Hölder's inequality. As \(\hat{f} \in L^1\cap L^2\)
    Thus, \(f_n \in L^1 \cap L^2\) as required. Now, we need to show that \(\lim_{n \to \infty} \lVert f_n - f \rVert_2 = 0\).
    Notice, that for every \(t\in\RR\), the pointwise convergence in \(\RR\)  holds
    \begin{equation*}
         \lim_{n \to  \infty} \vert  f(t)-f_n(t) \vert=0,                                           \\
    \end{equation*}
    Also, we can dominate this sequence of real numbers by
    \begin{equation*}
        \left|f(t)-f_n(t)\right|^2 \leq 2\left(|f(t)|^2+\left|f_n(t)\right|^2\right) \leq 4|f(t)|^2
    \end{equation*}
    and \(4|f|^2 \in L^1(\mathbb{R})\), so it is Lebesgue-integrable and by the Lebesgue Dominated Convergence Theorem,
    \begin{equation*}
        \lim_{n \to \infty}\left\|f-f_n\right\|_2^2=\lim_{n \to  \infty} \int_{-\infty}^{\infty}\left|f(t)-f_n(t)\right|^2 \mathrm{d}t =\int_{-\infty}^{\infty}\left(\lim_{n \to  \infty}\left|f(t)-f_n(t)\right|^2\right) \mathrm{d}t =0
    \end{equation*}
\end{proof}
Let \(\hat{f}_n\) be the Fourier transform of \(f_n\). By the Parseval-Plancherel theorem, \(\lVert \hat{f}_n \rVert_2^2 = 2\pi \lVert f_n \rVert_2^2 \leq \infty  \),
so \(\{\hat{f}_n\}_{n=1}^{\infty }\) is a sequence of functions in \(L^2\). 
Let us now take \(n \geq m\) and consider
\begin{align*}
    \frac{1}{2\pi }\lVert \hat{f} _n - \hat{f} _m \rVert _2 &= \lVert f_n - f_m \rVert_2^2 = \lVert ( \mathbb{1}_{[-n,n]} - \mathbb{1}_{[-m,m]} ) f \rVert_2^2 
    = \lVert ( \mathbb{1}_{[-n,-m]} + \mathbb{1}_{[m,n]} ) f \rVert_2^2 \\
    &\leq \lVert \mathbb{1}_{[-n,-m]} f \rVert_2 + \lVert \mathbb{1}_{[m,n]} f \rVert_2
    \leq \lVert \mathbb{1}_{[-\infty ,-m]} f \rVert_2 + \lVert \mathbb{1}_{[m,\infty ]} f \rVert_2 \xrightarrow[m\to \infty ]{} 0
\end{align*}
where the first equality is again due to the Parseval-Plancherel theorem.
Thus, \(\{\hat{f} _n\}_{n=1}^{\infty}\) is a Cauchy sequence in \(L^2\), and since \(L^2\) is complete, it converges to some
function \(\hat{f} \in L^2\). We now \textbf{define} the Fourier transform of \(f\in L^2\) as the limit of this sequence,
i.e.
\begin{equation}
    \hat{f} \equiv  \lim_{n \to \infty} \hat{f} _n = \lim_{n \to \infty} \int_{-\infty}^{\infty} f_n(t) e^{-itx} \mathrm{d}t
\end{equation}  
It is no longer an ordinary Lebesgue integral, but an \(L^2\)-limit of improper Riemann integrals. Of course, in case
of Lebesgue integrable functions, the two definitions coincide. 

All this work was to convince ourselves that it actually \textit{makes sense} to talk about the Fourier transform of
\(\sin (x)/x\), which is not Lebesgue integrable. Now, with clear conscience, we can calculate it.
