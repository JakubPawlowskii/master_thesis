\chapter{Spin transport in long-range anisotropic Heisenberg model\label{chap:spin_transport}}
\thispagestyle{chapterBeginStyle}

After all the technicalities of the previous chapter, we are finally ready to study spin transport in
the long-range anisotropic Heisenberg model. The main motivation of this investigation is the fact that this model admits 
two limits exhibiting ballistic spin transport (cf. Fig.~\ref{fig:spin_transport}), namely the free particles
with nearest-neighbor hopping for \(\alpha\to \infty,\; \Delta = 0\) and (in the thermodynamic limit) the Haldane-Shastry
model for \(\alpha = 2\)~\autocite{Haldane1988,Shastry1988}. In both of these limiting models the spin current 
\(j^{\sigma}\) commutes with the Hamiltonian and thus is strictly conserved. Thus, even though
the Hamiltonian~\eqref{eq:long_range} is not integrable for arbitrary \(\alpha\), one could suspect
that it will be \textit{nearly integrable}, in the sense described in the introduction, and support
interesting transport properties. Using spin density expansion and linear response theory of optical conductivity,
in this chapter we will show that this is indeed the case and the spin transport is \textit{quasibalistic} along a sharp
line in the parameter space \(\Delta = \exp(- \alpha + 2)\), which continuously connects the two limiting
cases mentioned above.

\begin{figure}[htbp]
    \centering
    \begin{tikzpicture}[scale=1.5]
      \colorlet{col1}{blue!30}
    
     \begin{scope}[smooth,draw=gray!20,y=0.3989422804cm]
          \filldraw[fill=col1] plot[id=f1, domain=-1.5:1.5,samples=100] function {6*exp(-6*x*x)};
          \draw[black] plot[id=f2,domain=-1.5:1.5,samples=100]
              function {6*exp(-6*x*x)};
      \end{scope}
      \draw[->] (-1.5,0) -- (1.5,0) node [right] {$x$};
      \draw[->] (-1.5,0) -- (-1.5,2.5) node [midway,rotate=90,yshift=6pt] {spin density};
      \draw[<->] (-0.37,1) -- (0.37,1) node [midway, yshift=6pt] {$\gamma$};
    
      \begin{scope}[smooth,draw=gray!20,y=0.3989422804cm]  
        \draw[black] plot[id=f3,domain=2.35:5.35,samples=100] function {0.7};    
        \draw[black] plot[id=f4,domain=2.35:5.35,samples=100] function {x};    
        \draw[black] plot[id=f5,domain=2.35:5.35,samples=100] function {sqrt(x)};    
      \end{scope}  
      
      \draw[->] (2.35,0) -- (5.35,0) node [right] {$t$};
      \draw[->] (2.35,0) -- (2.35,2.5) node [right] {$\gamma$};
      \node[draw=none] at (3.35,1.6) {$\gamma \propto t$};
      \node[draw=none] at (3.45,0.95) {$\gamma \propto \sqrt{t}$};
      \node[draw=none] at (3.6,0.4) {$\gamma = const$};
      \node[draw=none] (localized) at (6,0.3) {localized};
      \node[draw=none,above of=localized,node distance=27pt] (diffusive) {diffusive};
      \node[draw=none,above of=diffusive,node distance=45pt] (ballistic) {ballistic};
    \end{tikzpicture}  
    \caption{Illustration of different types of transport. On the left panel, we have some initial spin
    density characterized by width \(\gamma\). On the right panel, we have the dependence of \(\gamma\)
    on time in three different cases.}
    \label{fig:spin_transport}
  \end{figure}
  
  \section{Spin density expansion}


