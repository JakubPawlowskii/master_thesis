\chapter{Summary}
\thispagestyle{chapterBeginStyle}


The overarching goal of the work presented in this thesis is to understand the properties of
quantum systems that are close to integrability, as they facilitate real-world implementation, while
still maintaining some of the desired properties of strictly integrable systems.
To aid this development, there is a strong need for a robust set of theoretical tools that one
could use for the investigation of such systems. As most of the sophisticated machinery developed
for integrable systems cease to work for the case of nearly integrable systems, one has
to resort to numerical methods. It is impossible to touch upon all of the aspects of this
topic in a short thesis, thus we had two concrete goals in mind.

The first one was to provide a comprehensive introduction to the numerical methods, beyond the simplest exact
diagonalization, that are used in the study of quantum many-body systems. To this end, we presented
the Krylov subspace methods, which are designed to leverage the sparsity of matrix representations of
physical observables, in order to speed up the computations. Starting from the very beginning, we 
introduced in detail the Arnoldi iteration, which is most often used to find the extremal eigenvalues
of general, non-hermitian matrices, even though it does so rather accidentally. We then moved on to
the case of hermitian matrices, where this procedure reduces to the well-known and versatile
Lanczos iteration. As an immediate application of the Lanczos iteration, beyond just the groundstate
computations, the so-called Krylov propagator was presented, which allows one to compute the time evolution of
a state vector, without the need to diagonalize the Hamiltonian. Finally, we described,
and partially derived, an approach to the correlation functions, based on the Krylov
propagator and the concept of Quantum Typicality, which allows one to compute the correlation functions
without carrying out the trace of all many-body states.

The second goal was to apply those methods to the study of the dynamics of the spin transport in the anisotropic
Heisenberg model with power-law interaction \(J(\alpha) = J/r^{\alpha}\), which is an important example of a
long-range interacting system. Motivated by experimental setups, we started by considering the
expansion of a domain-wall initial state and found a non-monotonic dependence of the
center of mass velocity on the anisotropy parameter. By considering maximal velocities,
we have found a line in the parameter space, \(\Delta_O(\alpha) = \exp(-\alpha + 2)\),
for which the velocity of spin domain wall expansion is similar to the case of free
NN particles, indicating short-time (high-frequency) ballistic transport. 
Next, we studied spin transport from the perspective of the linear response theory,
by considering the optical conductivity and its integral over a frequency window. 
It revealed a regime of quasi-ballistic transport, where the spectral sum
of the optical conductivity accumulated at frequencies below \(\Omega^{\ast}/J \sim
10^{-3}-10^{-2}\). As a result, along the line of optimal anisotropy, the spin
transport is ballistic for surprisingly long times, up to \(t \sim 1/\Omega^{\ast}\),
which is in agreement with the results of the domain-wall expansion. 
Finally, we have investigated the properties of the local, slowly relaxing observables
(LSROs), obtained using a numerical algorithm designed to find local integrals of
motion in integrable systems. For completeness, we have also provided a rather detailed
description of the algorithm itself, together with a possible upgrade, replacing
ED-based time averaging with Lanczos-based long-time correlation functions.
Unfortunately, the upgraded approach turned out to be unsuitable for the purpose of
this thesis, as matrices of long-range operators are no longer sparse and thus do not
benefit from the Lanczos iteration. Nevertheless, using the original algorithm, we
have found that the best LSROs can be understood in terms of
current-like operators, corresponding via the Jorda-Wigner
transformation to long-range fermionic hoppings. Furthermore, those operators were
found to decay the slowest precisely for the optimal anisotropy, obtained from
previous methods. 

\textcolor{red}{State clearly main result}

% Our recent study on the anisotropic Heisenberg model with power-law decaying interaction has shown the existence of a sharp line
% in parameter space, that interpolates between two integrable regimes and facilitates transient ballistic
% spin transport~\autocite{Mierzejewski2023}.
% Using the LIOM algorithm, we have identified
% the most conserved operators (LSROs) and found that the largest contributions to their stiffness come from current-like operators,
% corresponding via Jorda-Wigner transformation to long-range fermionic hoppings. Furthermore, one of those operators
% were found to be similar to the initially studied spin current (in the sense of having a significant overlap).

% To summarize this part, we emphasize that the core goal of this project is to understand the properties of finite
% systems close to integrability. This includes both systems with integrability-breaking perturbations and
% systems that we do not expect to be integrable, yet behave as if they would be close to some (possibly unknown)
% integrable point. Our strategy is based on an investigation of local integrals of motion and their relation
% to the dynamics of physically relevant observables.

