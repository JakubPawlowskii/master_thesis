\chapter{Summary}
\thispagestyle{chapterBeginStyle}


The overarching goal of the work presented in this thesis, serving as the background is to understand the properties of
quantum systems that are close to integrability, as they facilitate real-world implementation, while
still maintaining some desired properties of strictly integrable systems.
To aid this development, there is a strong need for a robust set of theoretical tools that one
could use for the investigation of such systems. As most of the sophisticated machinery developed
for integrable systems cease to work for the case of nearly integrable systems, one has
to resort to numerical methods. It is impossible to touch upon all the aspects of this
topic in a short thesis, thus we had two concrete goals in mind.

The first one was to provide a comprehensive introduction to the numerical methods, beyond the simplest exact
diagonalization, that are used in the study of quantum many-body systems. To this end, we presented
the Krylov subspace methods, which are designed to leverage the sparsity of matrix representations of
physical observables, in order to speed up the computations. Starting from the very beginning, we
introduced in detail the Arnoldi iteration, which is most often used to find the extremal eigenvalues
of general, non-hermitian matrices, even though it does so rather accidentally. We then moved on to
the case of hermitian matrices, where this procedure reduces to the well-known and versatile
Lanczos iteration. As an immediate application of the Lanczos iteration, beyond just the ground state
computations, the so-called Krylov propagator was presented, which allows one to compute the time evolution of
a state vector, without the need for diagonalization of the Hamiltonian. Finally, we described,
and partially derived, an approach to the correlation functions, based on the Krylov
propagator and the concept of Quantum Typicality, which allows one to compute the correlation functions
without explicitly carrying out the trace over all many-body states.

The second goal was to apply those methods to the study of the dynamics of the spin transport in the anisotropic
Heisenberg model with power-law interaction \(J(\alpha) = J/r^{\alpha}\), which is an important example of a
long-range interacting system. Motivated by experimental setups, we started by considering the
expansion of a domain-wall initial state and found a non-monotonic dependence of the
center of mass velocity on the anisotropy parameter. By considering maximal velocities,
we found a line in the parameter space, \(\Delta_O(\alpha) = \exp(-\alpha + 2)\),
for which the velocity of spin domain wall expansion is similar to the case of free
nearest-neighbors particles, indicating short-time (high-frequency) ballistic transport.
Next, we studied spin transport from the perspective of the linear response theory,
by considering the optical conductivity and its integral over a frequency window.
It revealed a regime of quasi-ballistic transport, where the spectral sum
of the optical conductivity accumulated at frequencies below \(\Omega^{\ast}/J \sim
10^{-3}-10^{-2}\). As a result, along the line of optimal anisotropy, the spin
transport is ballistic for surprisingly long times, up to \(t \sim 1/\Omega^{\ast}\),
which is in agreement with the results of the domain-wall expansion.
We also investigated the properties of local, slowly relaxing observables
(LSROs), obtained using a numerical algorithm designed to find local integrals of
motion in integrable systems. For completeness, we provided a rather detailed
description of the algorithm itself, together with a possible upgrade, replacing
ED-based time averaging with Lanczos-based long-time correlation functions.
Unfortunately, the upgraded approach turned out to be unsuitable for the purpose of
this thesis, as matrices of long-range operators are no longer sparse and thus do not
benefit from the Lanczos iteration. Nevertheless, using the original algorithm, we
found that the best LSROs decay the slowest precisely for the optimal anisotropy, obtained from
previous methods. Additionally, we showed that they can be understood in terms of projections onto
current-like operators, corresponding via the Jordan-Wigner transformation to long-range fermionic hoppings,
even though the Hamiltonian does not admit a representation in terms of two-body fermionic operators.
Finally, using finite-time averaging version of the algorithm, we investigated the frequency dependence
of the LSROs, akin to the optical conductivity, and found that the concentration of the spectral weight
resembles the one of the optical conductivity.

The main result of this thesis, and the broader research published in \autocite{Mierzejewski2023}, is the
discovery of the line of optimal anisotropy, \(\Delta_{O}(\alpha) = \exp(-\alpha +2)\),
along which the spin transport is ballistic for surprisingly long times \(tJ \sim 10^{-2} -10^{-3}\).
This line smoothly interpolates between integrable two models with purely ballistic
transport, namely free particles for \(\alpha =\infty,\; \Delta =0\) and Haldane-Shastry
like model for \(\alpha  = 2,\; \Delta  = 1\). It suggests that the long-range Heisenberg
model can be thought of as nearly-integrable far beyond just the vicinity of the
integrable limits, but rather for a whole line of parameters. Unfortunately, the precise
microscopic origin of the optimal anisotropy is still unknown. A possible explanation
would be that for \(\delta = \exp(-\alpha  + 2)\) this model is very close to some,
more complicated, integrable model, in which the spin current is an integral of motion.
Investigation of systems similar to the one studied in this thesis, from the perspective
of the search for remnants of integrability, is a promising direction for future research.
