\chapter{Introduction\label{chap:intro}}
\thispagestyle{chapterBeginStyle}

\section{Nearly integrable quantum systems}
One of the most widely disputed problems in modern physics is the reconciliation
of irreversible thermodynamics~\autocite{huang1987statistical,feynman1998statistical}
with unitary, time-reversible dynamics predicted by quantum mechanics~\autocite{Landau1976,Sakurai2017}.
In other words, it is the question of whether a generic, isolated quantum system can and will `forget' about its
initial, nonequilibrium state. In recent years, this problem has attracted the attention of many physicists,
especially taking into account experimental evidence of the loss of information, or thermalization, in isolated
systems~\autocite{Trotzky2012325,Rigol2012,Rigol2008854,Hung2010,Hofferberth2007}. Understanding this phenomenon is a crucial first step to
controlling it, and perhaps to the creation of sought-after, robust systems which do not exhibit such
behaviour. After all, `forgetting' about the initial state is equivalent to scrambling of quantum information
and decoherence, which has been known for a long time to be one of the most important hindrances 
to a functional and scalable quantum computer~\autocite{Shor1995,LewisSwan2019}. Given the multitude
of possible applications of such a device, both purely scientific and commercial, controlling thermalization
in isolated systems could be the next groundbreaking achievement~\autocite{MacQuarrie2020}.
 
\paragraph{Preventing thermalization} A possible theoretical explanation of the thermalization in isolated quantum systems
has been proposed in seminal papers by~\textcite{Deutsch19912046,Srednicki1994} in form of an ansatz for matrix elements of 
local quantum observables \(A_{mn} = \matrixel{m}{A}{n}\) (in eigenbasis \(H\ket{n}=\varepsilon_n \ket{n}\) of some Hamiltonian),
known as \textbf{Eigenstate Thermalization Hypothesis} (ETH)
\begin{equation}
    A_{mn} = A(\bar{\varepsilon}) \delta_{mn} + \mathrm{e}^{-S(\bar{\varepsilon})/2}f_{A}(\bar{\varepsilon},\omega)R_{mn}
    \label{eq:ETH}
\end{equation}
where \(\bar{\varepsilon} = (\varepsilon_n+\varepsilon_m)/2\), \(\omega=\varepsilon_n-\varepsilon_m\) and \(S(\bar{\varepsilon})\) is
the thermodynamic entropy at energy \(\bar{\varepsilon}\),  \(A(\bar{\varepsilon})\) and \(f_{A}(\bar{\varepsilon},\omega)\) are
some smooth functions of their parameters and
\(R_{mn}\) is a random real or complex variable. The origin of ETH is the Random Matrix Theory~\autocite{Wigner1955548,mehta2004random},
but it was the key insight of Deutsch and Srednicki that connected it directly to statistical mechanics.
Unraveling the physics behind equation~\eqref{eq:ETH}, we have that, for generic systems,
long-time (thermalized) expected values of local observables are related to the predictions of Gibbsian ensembles
known from equilibrium statistical physics~\autocite{DAlessio2016}. Diagonal matrix elements \(A_{nn} \simeq A(\varepsilon_n)\) are smooth functions of energy and
coincide with the microcanonical average at energy \(\varepsilon_n\), whereas the off-diagonal matrix elements and fluctuations
of the diagonal ones are postulated to decrease exponentially with system size~\autocite{Beugeling2014}.
While ETH has been shown to hold analytically in only a few selected systems~\autocite{Magan2016}, numerical clues can be
found in various systems with many-body interactions such as hard core bosons, interacting spin 
chains~\autocite{Santos2010_2,Rigol2010,Khatami2013,Rigol2009a} and fermions~\autocite{Neuenhahn2012,Rigol2009}.

Given the lack of a definitive answer to the question of when ETH holds and the fact that our 
ultimate goal is preventing thermalization, it is desirable to look for systems
which explicitly violate this hypothesis. There are a few known classes of such systems:
quantum integrable models, systems with `scars', which freeze thermalization for
some infrequent, but physically relevant states~\autocite{Turner2018a,Turner2018}, quantum time crystals with
broken time-translation symmetry~\autocite{Wilczek2012,Sacha2017} and systems with many-body interactions together
with some form of disorder~\autocite{Basko_2006}. As the thermalization in those systems is slowed down or even
completely stopped, in principle they could preserve information about the initial state for an arbitrarily long time. 
This behaviour is in stark contrast with what is usually observed in interacting, many-body systems, namely fast
dynamics on the time scales of tens of femtoseconds~\autocite{DalConte2015}.
In this thesis we shall concern ourselves only with the case of \textbf{quantum integrable models}.

\paragraph{Robustness of integrability} 
In quantum mechanics, as opposed to classical mechanics, a unique definition of integrability
is yet to emerge~\autocite{Caux2011,Yuzbashyan2013}. Nevertheless, a common practice is to take as a sufficient condition of
integrability the presence of an extensive number (increasing linearly with the system size) of local observables
that are integrals of motion, i.e. commute with the Hamiltonian, called LIOMs for short. The significance of such
systems is at least twofold. 
First, as already mentioned, they can provide a new method of storage of information, encoded in expectation values
of LIOMs, which can pave the way for the creation of the sought-after quantum memory.
Second, they are among a few quantum many-body systems that are susceptible to analytical methods and so provide
a rich playground for both theoretical physicists and mathematicians. Tools such as Algebraic Bethe
Ansatz~\autocite{Faddeev1995,Faddeev1996,Korepin1993} or Generalized Hydrodynamics~\autocite{Agrawal2020,Friedman2020,
Bertini2021,Bastianello2021,Bulchandani2021} lead to considerable insight regarding the nature of integrable systems.

Unfortunately, it is often the case for integrable systems such as Heisenberg, Hubbard or Lieb-Linger
models, that their integrability relies on a certain set of fine-tuned parameters. Any deviation from these parameters, eg.
in the form of some perturbation, can very easily destroy the integrability. Exactly this situation takes place in most
experimental setups (albeit some signatures of integrability were observed~\autocite{Khemani2019}, suggesting sufficient
proximity to an integrable system),  even though the capabilities and precision of control have
reached remarkable levels~\autocite{Browaeys2020}. Therefore, \textbf{more realistic systems
are expected to be described by nearly integrable systems}, containing some non-negligible perturbations that
impact integrability~\autocite{Zotos2004,Brandino2015}.
This renders all the formalism for investigations of purely
integrable systems inapplicable and forces one to rely mostly on numerical methods (for a recent review
see~\textcite{Bertini2021}).
A general expectation for such nearly integrable systems is such, that in the thermodynamic limit, an
arbitrarily small perturbation should restore generic chaotic dynamics, which leads to thermalization
~\autocite{LeBlond2021}. However, there
are some reports about surviving traces of integrability, for example in the form of residual quasiconserved
quantities~\autocite{Brandino2015}. In most cases numerical methods can only access finite 
systems and indirect transitions to infinite system sizes, such as finite-size scaling, pose great difficulties
when executed properly, eg. first thermodynamic limit and then infinite time limit~\autocite{Sirker2014,CamposVenuti2010}.
Therefore, it is important to first gain a deep understanding of \textbf{weakly-perturbed finite systems}. Such
endeavour immediately raises a few questions: How strong should a perturbation in a finite
systems be to destroy integrability and how to efficiently describe resulting slow dynamics? 
Recently, those and other similar questions have been asked concerning the phenomena of ergodicity breaking
phase transitions~\autocite{Suntajs2020,Suntajs2022}.
Insight from experiments with ultracold atomic gases suggest also another way for nearly integrable systems to emerge,
namely generic, chaotic one- and quasi-one dimensional systems that exhibit approximately integrable dynamics on experimentally relevant timescales,
because of a lack of rapid local thermalization~\autocite{Polkovnikov2011863,Gopalakrishnan2023}.

In classical mechanics, answers to those types of questions are given by the Hamiltonian perturbation
theory~\autocite{arnold2013mathematical}. There, the distinction between integrable and non-integrable systems is
well understood. It is known, that for classical systems with \(n\) degrees of freedom
to be integrable, it is sufficient to have \(n\) integrals of motion \(\{H,F_i\} = 0,\; i \in \{1,\ldots,n\} \) that
are in involution, i.e. \(\{F_i,F_j\} = 0\) for any \(i,j\in\{1,\ldots,n\}\). Then, dynamics are restricted to an \(n\)-dimensional
torus in the phase space. The fate of such invariant tori under integrability-breaking perturbations is given by the
famous Kolmogorov-Arnold-Moser theorem (KAM)~\autocite{Kolmogorov1954,Arnold1963,Moser1962}, stating that for systems
with finite degrees of freedom, majority of the invariant tori occupying the phase space survive the influence of small 
perturbations. However, as of now, there is no equivalent of this result in quantum mechanics.
Nevertheless, such quantum systems are very intriguing as they can facilitate robust
prethermalization plateaux, i.e. dynamics that at intermediate times resemble that of integrable models
(at least for suitably weak perturbations), even though the system eventually thermalizes at longer
times~\autocite{mallayya2019,Bertini2015,Berges2004,Langen2016}.

\section{Motivation and aim of this dissertation}
As argued in the previous section, nearly integrable quantum systems constitute an important relaxation
of constraints imposed on ordinary integrable systems, facilitating experimental realizations. They form the broader context this
thesis is set in and the main motivation. However, to keep it finite in size, we restrict our attention
to two concrete problems. 

First, we provide a pedagogical introduction to the so-called \textbf{Krylov subspace methods} which allow
us to leverage the sparsity of local observables
in order to avoid the limitations originating from exponential growth of many-body Hilbert space.
Following the exposition by~\textcite{Trefethen1997}, in Chapter~\ref{chap:krylov} we start our journey seemingly far away from
physics, investigating a general algorithm called Arnoldi iteration, which original aim was to reduce
a matrix to `almost triangular' or Hessenberg form. Along the way we discover that, as a byproduct,
Arnoldi iteration produces a remarkably good approximations of extremal eigenvalues and eigenvectors. Contrary
to most physics texts and in spirit of pedagogical nature of this chapter, we try to derive all results when possible
and motivate them thoroughly when not. Setting course back to physics, next we assume that our matrices are
Hermitian and observe how the Arnoldi interation simplifies tremendously, producing the well known
Lanczos iteration~\autocite{Sandvik2010}. Then, going beyond just groundstate calculations, we show how to slighlty
modify Lanczos iteration to be able to compute an approximation of action of any analytic function of hermitian
matrix on a vector. Applying this scheme to the function \(f(x) = \exp\left(-i x t\right)\), we obtain an efficient
way of calculating pure state time evolution, called the Krylov propagator~\autocite{Park1986}.
At the end of Chapter~\ref{chap:krylov}, we outline the recent concept of Quantum Typicality~\autocite{Bartsch2009}
and derive a numerical procedure for efficient calculation of correlation functions, without the need for Exact
Diagonalization.

\textcolor{red}{Go back to this part after finishing chapters on physics, add some citations}
Our second topic of interest is the \textbf{spin transport in the long range anisotropic Heisenberg model.}
Existence of many interesting features of quantum many-body systems, such as ballistic dynamics in Heisenberg spin
chains~\autocite{Zotos1997,Bertini2021},
exotic frustrated magnetism~\autocite{Nisoli2017}, peculiar phase transitions~\autocite{Sandvik2010b,Yang2021}
and entangled spin liquids~\autocite{Balents2010} depends strongly
on the type and range of interaction present in them, so it is safe to say that it plays a crucial role.
In the last few years, considerable development of experimental techniques has
occurred, allowing for unprecedented manipulation of the interactions.
Platforms such as individually controlled Rydberg atoms, or optical lattices provide insight into the
properties of quantum many-body systems. Whereas optical lattice facilitates mostly fermionic 
systems~\autocite{Bakr2009,Greif2016,Parsons2015,Boll2016}, Rydberg atoms can be used to simulate pure spin systems.
Models such as Ising or XY emerge naturally from their properties~\autocite{Browaeys2020}, and the ability to control
range of interactions make them suitable for the simulation of long-range models~\autocite{Borish2020}. Using time-periodic
driving one can turn naturally existing Hamiltonian into some other, effective one - the so-called Floquet Hamiltonian.
So far, this method has been applied with great success to create tunable XXZ model~\autocite{Scholl2022}, strongly distance
selective interactions~\autocite{Hollerith2022} and tunable XYZ models~\autocite{Steinert2022, Geier2021}.
Progress in experimental methods sparked theoretical interest in long-range
models~\autocite{Richerme2014,Jurcevic2014,Hauke2013,FossFeig2015,Maghrebi2016,Lepori2017,Frerot2017,Vanderstraeten2018,Cevolani2018,Kloss2019,Ren2020,Bulchandani2022a},
yet their dynamical properties are still largely unknown, which explains our interest in long range anisotropic
Heisenberg model in this thesis.
In Chapter~\ref{chap:spin_transport}, motivated by experiments with density expansion in cold atoms~\autocite{Ronzheimer2013,Vidmar2013,Neyenhuis2017},
we study the dynamics of spin domains using the Krylov propagator, followed by Exact Diagonalization studies of the optical conductivity.
Chapter~\ref{chap:currents} is devoted to investigating a class of local observables exhibitng similar properties
to the spin current, using a numerical procedure searching for most conserved operators~\autocite{Mierzejewski2015a}.
It is worth noting that the results presented in Chapters~\ref{chap:spin_transport} and~\ref{chap:currents} have
been recently published in~\textcite{Mierzejewski2023}.

The remainder of this chapter is devoted to the short introdction to the long range anisotropic Heisenberg model
and other quantities of interest.

\section{Long range anisotropic Heisenberg model and spin current}

In this thesis we study the paradigmatic quantum model of magnetism, anisotropic Heisenberg
model, but enriched with a long-range exchange \(J(r) = J/r^{\alpha}\). The full Hamiltonian, 
defined on a one-dimensional lattice with periodic boundary conditions, reads
\begin{equation}
    H = \sum_{\ell=1}^{L}\sum_{r=1}^{r_{\mathrm{max}}} J(r) \left[\frac{1}{2} \left(
        \Sp_{\ell}\Sm_{\ell+r} + \Sm_{\ell}\Sp_{\ell+r}\right) + \Delta \Sz_{\ell}\Sz_{\ell+r}
    \right]
    \label{eq:long_range}
\end{equation}
where \(r_{\mathrm{max}}\) is taken to be \(\left\lceil L/2\right\rceil - 1 \), to avoid double counting of hoppings.

